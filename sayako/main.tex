\documentclass{article}
\usepackage{xeCJK}
\usepackage{amssymb}
\usepackage{amsthm}
\usepackage{mathtools}
\usepackage{amsmath}
\usepackage{amsfonts}
\usepackage{enumitem}
\usepackage{unicode-math}
\usepackage[a4paper, left=2.5cm, right=2.5cm, top=1.25cm, bottom=1.25cm]{geometry}

\setCJKmainfont[BoldFont={SimHei}]{SimSun}
\setCJKsansfont{SimHei}
\setCJKmonofont{FangSong}
\setlength{\parindent}{0cm}
\setlength{\parskip}{1em}

\title{Sayako's Story}
\author{Yutong "Hypatia" "Sayako" Zhang}
\begin{document}

\maketitle
\setcounter{section}{-1}
\section{前言}
这是名为Sayako的跨性别女孩的故事,也就是我的故事。写在这里希望能够被看到和铭记,如果有一天我离开了这个世界的话,希望还有人能够记得她——Sayako、她的故事,一个美丽而坚强的数学女孩的故事、和她生命中的重要的人——Wendy,她的姐姐,Steven,她的初恋。

本文将尽量用中文书写,但对话中有一些只能用英语写出的内容和表述,这些表述将直接用英文书写。

\section{正弦的幂次}
2017年9月某日

我像往常一样独自一人坐在中学教学楼3层西边的角落的沙发上,在来到这个学校的一个月中我没有交到任何朋友——男生们会对我开令我很不适的玩笑(我不觉得对一个陌生人开黄腔是一件多么好的事情,尤其是对一个陌生的异性——至少我自认为他们是异性),而女生们总对我抱有距离感,使得我每次尝试分享自己的感受都不了了之。

然后出现了一个奇怪的人,他留着十分夸张的卷发,肤色颇深,他走到我身边,看起来对我正在写的Coq证明很感兴趣(我也很喜欢理论计算机科学,为此学习了编程语言理论和计算机辅助证明)。

“你听说过calculus of inductive construction嘛?”我尝试搭话,事实上根据他的外貌我还不确定他会不会说中文,所以这句话其实是用英文说的。

“我知道calculus。你会calculus嘛?”他似乎兴奋地回答。

嘛……我们说的calculus根本不是同样的东西好不好,我暗自吐槽道。

“数学分析可是基本功哦$\sim$”我尝试这自认为可爱的语气,希望能够给这个可能成为朋友的男生一点好印象(但是后来事实证明在我transition之前这样卖萌的语气根本就是减分,男生果真不喜欢卖萌嘛qwq)。

“听说我们明年就会学到calculus哦。”他并没有注意到我的心思,继续说道。

事实上我已经过了一遍Hasse Publication的那本IB DP Math HL的书了,评价是……怎么说呢,闭口不谈$\varepsilon-\delta$语言,明明是基本功啊喂,这样好嘛……不过其实倒也和数学分析没差多少,毕竟(在闭口不谈$\varepsilon-\delta$语言的情况下)在一道练习中提到了用数列刻画函数的连续性和Dirichlet函数的性质还有$\exp$的级数定义什么的(所以大概函数项级数也有涉及?),不过和我学的Rudin以及听说过的Zorich等书还是有一定的差距,但是总之比高考的导数好多了(又黑高考)。

“好呀,既然你提到了calculus,那我就来考考你,”我突然有一种想要调戏面前男生的冲动,于是由此提议。

“你听说过$\sin$的幂次的不定积分嘛?”我决定出一道经典习题(大概是吉米多维奇上有的吧……)\footnote{对这道题的答案感兴趣的读者可以查看http://www.ijapm.org/vol5/336-WP1005.pdf},拿出草稿纸写下
$$
\int\sin^nax=?\quad n\in\mathbb{Z}_{>0},a\in\mathbb{R}\mathbin{\backslash}\{0\}
$$

“给你个提示哦,”我接着写下去,“尝试证明这个等式,然后使用这个等式。”
$$
\int \sin ^{n} a x d x=-\frac{1}{an} \sin ^{n-1} a x \cos a x+\frac{n-1}{n} \int \sin ^{n-2} a x d x
$$

经过漫长的尝试,我发现他并没有很好地掌握分部积分法,遂放弃让他解出这个题目的尝试。

“话说你叫什么呢?”我发现这么长时间我还没有知道对方的名字,于是问道。

“叫我Steven就好。”

这就是我和他相识的故事。

\section{阴道独白}
2017年10月某日

\section{一致收敛}
2018年1月某日

“estuviera,estuviese,estuvieras,estuvieses,……,”现在是下午2点左右,再过十分钟我就要去上我的第二语言课程IB MYP Spanish Phase 1了,我正在看Wiktionary上“estas”这个系动词的变位形式,希望能够记下来它的subjunctive形式,尽管事实上还要一年这门课程才会讲到这里。

“Yutong,你不应该这样做。你知道我不喜欢你们背诵变位,我希望你们能够learn by chunks。”

这是我的Spanish老师,其实我很不喜欢language acquisition by immersion的理论,但是从这一个学期的观察我觉得这个老师的确非常into这个理论。

“我应该说‘portiempo, quiero que yo estuviera una chica’还是‘portiempo, quiero que yo estaba una chica’,这里有两个相同的主语?”我飞快地问了一个问题。

“estuviera, with el subjuntivo,”老师还是回答了我的问题,"come inside, boys, it's time.”老师招呼大家进入教室。

我其实并不喜欢被叫做boys,但是从老师的角度看的确面前的学生都是男生,唯一的一个女生已经坐进了教室,我也不好张口反驳。如果我是女孩子就好了。

“Okay, ten minutes for you to socialize, Yutong off you go.”这个老师每次开始上课之前总是给我们时间聊聊今天发生的事情什么的,她也喜欢听学生们讲他们身上发生的事情。

“我觉得我找到了一个Mr. Patrick的错,他数学课上随口说了一句‘不管有多少项,先积分再求和和先求和再积分的结果总相同’,但是如果有无限多项的话,需要这个级数是一致收敛才可以交换积分号和求和号。”我很喜欢数学课和数学,所以基本上每次让我分享的时候我都会提及数学课上发生的趣事。

“Ah, that's our math boy.”

别再叫我boy了哇!

“You're showing off, Yutong.”说话的是Alex,和我同上Spanish的同学。

“To whom do you think I'm showing off? ”不知道他有多么讨厌数学——明明是那么美丽的学科,他每次都要这样嘲讽我一番,今天也不例外,不过这次我回击了一句。

“To the girls, adolescent boy!”我知道他想说什么了,好像所有人都只是想要让异性对自己感兴趣一样,事实上只有他自己这样想吧(后来我了解到这叫做heteronormativity,以及大部分青春期男生的确是这样想的),然后我决定不再理他。

“看吧看吧果真就是,”这句话他用中文说的,似乎并不想让老师听懂,虽然对这种事情老师大概也是喜闻乐见就是了。

嗯……虽然我当时并没有打算那么做,但是如果可以教我喜欢的男孩子数学或被他教数学的话好像还是很浪漫的事情(事实是他教我的是化学,而且只有一次,我们还不是情侣关系,执念w),不过要在独处的时候,在课堂上算什么哇,秀恩爱嘛qwq。

这件事情让我记忆很深(以至于我都记得我当时背的词是哪几个)也是因为Alex,我清楚地记得后来我了解到男生之间的交流方式并明确是被当作男士时是如何地无奈。

其实仔细回想起来我从不想其他长辈说的那样“谦虚一点”而是直接表露自己的想法也可能是因为嫉妒,我想要变得比他们(顺性别者)更加优秀,并且让他们明确地知道我更加优秀,以此来掩饰自己羡慕他们的生活这一点。真的好羡慕女孩子哦,也羡慕不是药娘的男孩子,他们能够热爱自己的指派性别和身体,接受他人对他们masculinity的称赞就像我接受别人对我feminity的称赞一样,他们的生活一定都比我顺利吧。

\section{阿贝尔群}
2018年1月某日

\end{document}

